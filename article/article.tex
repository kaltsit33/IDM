\documentclass{aa}
\usepackage{graphicx}
\usepackage{txfonts}

\begin{document} 

   \title{Self-interacting dark matter model without dark energy in cosmology}

   \author{Yixuan Zhu}

   \institute{Department of Astronomy, Beijing Normal University.
   Beijing 100875.
   PR China}

   \date{}
 
   \abstract{}
   
   \keywords{}

   \maketitle

\section{Introduction}

\section{The basic equations in the IDM model}

   We assume that the total density of the cosmic fluid obeys
   the collisional Boltzmann equation()
   \begin{equation}
      \dot{\rho}+3H\rho+\kappa\rho^2-2\Psi=0,
   \end{equation}
   where $\rho$ is the total energy-density of the cosmic fluid,
   containing dark matter, baryons, and any type of exotic energy,
   $\Psi$ is the rate of creation of DM particle pairs, and the
   annihilation parameter $\kappa(\geq0)$ is given by:
   \begin{equation}
      \kappa=\frac{\langle\sigma u\rangle}{M_x},\label{eq:2}
   \end{equation}
   where $\sigma$ is the cross-section for annihilation, $u$ is
   the mean particle velocity, and $M_x$ is the mass of the DM
   particle.
   Compared to the usual fluid equation, the effective pressure term
   is \begin{equation}
      P=\frac{\kappa\rho^2-\Psi}{3H}.
   \end{equation}
   When $\kappa\rho^2-\Psi<0,$ what means that the IDM particle
   creation term is larger than the annihilation item, IDM may serve
   as a negative pressure source in the global dynamics of the Universe,
   like the role of Dark Energy in the general cosmolgical models.

   Basilakos \& Plionis (2009) identified two functional forms for which
   the previous Boltzmann equation can be solved analytically.
   Refering to Appendix B in (), only one of these two is of interest
   because it provides a "$\propto a^{-3}$" dependence of the scale factor,
   which is \begin{equation}
      \Psi(a)=aH(a)R(a)=C_1(n+3)a^nH(a)+\kappa C_1^2a^{2m}.
   \end{equation}
   And the total energy density is
   \begin{equation}
      \rho(a)=C_1a^n+\frac{a^{-3}F(a)}{C_2-\int_1^{a}x^{-3}f(x)F(x)\mathrm{d}x},\label{eq:5}
   \end{equation}
   where $f(a)=-\kappa/[aH(a)],$ and the kernal function $F(a)$ has
   the form \begin{equation}
      F(a)=\exp\left[-2\kappa C_1\int_1^{a}\frac{x^{n-1}}{H(x)}\mathrm{d}x\right].
   \end{equation}
   The first term of Eq.(\ref{eq:5}) is the density corresponding to the
   residual matter creation that results from a possible disequilibrium
   between the particle creation and annihilation processes, while the second
   term can be viewed as the energy density of the self-IDM particles that are
   dominated by the annihilation process.

\subsection{Model 1: relation to the $\varLambda$CDM model}

   If $n=0$, the global density evolution can be transformed as
   \begin{equation}
      \rho(a)=C_1+a^{-3}\frac{e^{-2\kappa C_1(t-t_0)}}{C_2-\kappa Z(t)},
   \end{equation}
   where $Z(t)=\int_{t_0}^ta^{-3}e^{-2\kappa C_1(t'-t_0)}\mathrm{d}t'$().
   Using the usual unit-less $\Omega$-like parameterization, we obtain that
   \begin{equation}
      \left(\frac{H}{H_0}\right)^2=\Omega_{1,0}+\frac{\Omega_{1,0}\Omega_{2,0}a^{-3}e^{-2\kappa C_1(t-t_0)}}
      {\Omega_{1,0}+\kappa C_1\Omega_{2,0}Z(t)},
   \end{equation}
   where $\Omega_{1,0}=8\pi GC_1/3H_0^2$ and $\Omega_{2,0}=8\pi G/3H_0^2C_2$,
   which related to $\Omega_{\Lambda}$ and $\Omega_m$ in the $\Lambda$CDM model,
   respectively.
   From Eq.({\ref{eq:2}}), we can also give the mass of the DM particle
   related to the range of $\kappa C_1$(in the unit of Gyr${}^{-1}$)
   \begin{equation}
      M_x=\frac{3.325\times10^{-12}}{\kappa C_1}
      \frac{\langle\sigma u\rangle}{10^{-23}}h^2(1-\Omega_{2,0})\,\text{GeV},
   \end{equation}
   where $h\equiv H_0/[100\text{km/s/Mpc}]$.

\subsection{Model 2 : relation to the wCDM model}

   If $\kappa=0$, the global density evolution can be written as
   \begin{equation}
      \rho(a)=\mathcal{D}a^{-3}+C_1a^{n},
   \end{equation}
   where $\mathcal{D}=C_2-C-1$. The conditions in which the current 
   model acts as a quintessence cosmology are given by $\mathcal{D}>0,
   C_1>0,$ and $w_{\text{IDM}}=-1-n/3$. This solution is mathematically
   equivalent to that of the gravitational matter creation model of().
   The Hubble flow is now given by
   \begin{equation}
      \left(\frac{H}{H_0}\right)^2=\Omega_{2,0}a^{-3}+\Omega_{1,0}a^{n},
   \end{equation}
   where $\Omega_{2,0}=8\pi G\mathcal{D}/3H_0^2$ and 
   $\Omega_{1,0}=8\pi GC_1/3H_0^2$, respectively.()
   
\section{Observational data}

\section{Constraint results}

\section{Conclusions}

\end{document}