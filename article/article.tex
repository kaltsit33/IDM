\documentclass{aa}
\usepackage{graphicx}
\usepackage{txfonts}
\graphicspath{{./pictures/}}

\begin{document} 

   \title{Self-interacting dark matter model without dark energy in cosmology}

   \author{Yixuan Zhu}

   \institute{Department of Astronomy, Beijing Normal University.
   Beijing 100875.
   PR China}

   \date{}
 
   \abstract{}
   
   \keywords{}

   \maketitle

\section{Introduction}

\section{The basic equations in the IDM model}

   We assume that the total density of the cosmic fluid obeys
   the collisional Boltzmann equation
   \begin{equation}
      \dot{\rho}+3H\rho+\kappa\rho^2-2\Psi=0,
   \end{equation}
   where $\rho$ is the total energy-density of the cosmic fluid,
   containing dark matter, baryons, and any type of exotic energy,
   $\Psi$ is the rate of creation of DM particle pairs, and the
   annihilation parameter $\kappa(\geq0)$ is given by:
   \begin{equation}
      \kappa=\frac{\langle\sigma u\rangle}{M_x},\label{eq:2}
   \end{equation}
   where $\sigma$ is the cross-section for annihilation, $u$ is
   the mean particle velocity, and $M_x$ is the mass of the DM
   particle.
   Compared to the usual fluid equation, the effective pressure term
   is \begin{equation}
      P=\frac{\kappa\rho^2-\Psi}{3H}.
   \end{equation}
   When $\kappa\rho^2-\Psi<0,$ what means that the IDM particle
   creation term is larger than the annihilation item, IDM may serve
   as a negative pressure source in the global dynamics of the Universe,
   like the role of Dark Energy in the general cosmological models.

   \cite{refId0} identified two functional forms for which
   the previous Boltzmann equation can be solved analytically.
   Refering to Appendix B in \cite{refId0}, only one of these two is of interest
   because it provides a "$\propto a^{-3}$" dependence of the scale factor,
   which is \begin{equation}
      \Psi(a)=aH(a)R(a)=C_1(n+3)a^nH(a)+\kappa C_1^2a^{2m}.
   \end{equation}
   And the total energy density is
   \begin{equation}
      \rho(a)=C_1a^n+\frac{a^{-3}F(a)}{C_2-\int_1^{a}x^{-3}f(x)F(x)\mathrm{d}x},\label{eq:5}
   \end{equation}
   where $f(a)=-\kappa/[aH(a)],$ and the kernal function $F(a)$ has
   the form \begin{equation}
      F(a)=\exp\left[-2\kappa C_1\int_1^{a}\frac{x^{n-1}}{H(x)}\mathrm{d}x\right].
   \end{equation}
   The first term of Eq.(\ref{eq:5}) is the density corresponding to the
   residual matter creation that results from a possible disequilibrium
   between the particle creation and annihilation processes, while the second
   term can be viewed as the energy density of the self-IDM particles that are
   dominated by the annihilation process.

\subsection{Model 1: relation to the $\varLambda$CDM model}

   If $n=0$, the global density evolution can be transformed as
   \begin{equation}
      \rho(a)=C_1+a^{-3}\frac{e^{-2\kappa C_1(t-t_0)}}{C_2-\kappa Z(t)},
   \end{equation}
   where $Z(t)=\int_{t_0}^ta^{-3}e^{-2\kappa C_1(t'-t_0)}\mathrm{d}t'$(\cite{refId0}).
   Using the usual unit-less $\Omega$-like parameterization, we obtain that
   \begin{equation}
      \left(\frac{H}{H_0}\right)^2=\Omega_{1,0}+\frac{\Omega_{1,0}\Omega_{2,0}a^{-3}e^{-2\kappa C_1(t-t_0)}}
      {\Omega_{1,0}+\kappa C_1\Omega_{2,0}Z(t)},\label{eq:8}
   \end{equation}
   where $\Omega_{1,0}=8\pi GC_1/3H_0^2$ and $\Omega_{2,0}=8\pi G/3H_0^2C_2$,
   which related to $\Omega_{\Lambda}$ and $\Omega_m$ in the $\Lambda$CDM model,
   respectively.
   From Eq.({\ref{eq:2}}), we can also give the mass of the DM particle
   related to the range of $\kappa C_1$(in the unit of Gyr${}^{-1}$)
   \begin{equation}
      M_x=\frac{3.325\times10^{-12}}{\kappa C_1}
      \frac{\langle\sigma u\rangle}{10^{-23}}h^2(1-\Omega_{2,0})\,\text{GeV},\label{eq:9}
   \end{equation}
   where $h\equiv H_0/[100\text{km/s/Mpc}]$.

\subsection{Model 2 : relation to the wCDM model}

   If $\kappa=0$, the global density evolution can be written as
   \begin{equation}
      \rho(a)=\mathcal{D}a^{-3}+C_1a^{n},
   \end{equation}
   where $\mathcal{D}=C_2-C-1$. The conditions in which the current 
   model acts as a quintessence cosmology are given by $\mathcal{D}>0,
   C_1>0,$ and $w_{\text{IDM}}=-1-n/3$. This solution is mathematically
   equivalent to that of the gravitational matter creation model of().
   The Hubble flow is now given by
   \begin{equation}
      \left(\frac{H}{H_0}\right)^2=\Omega_{2,0}a^{-3}+\Omega_{1,0}a^{n},
   \end{equation}
   where $\Omega_{2,0}=8\pi G\mathcal{D}/3H_0^2$ and 
   $\Omega_{1,0}=8\pi GC_1/3H_0^2$, respectively.(\cite{refId0})
   
\section{Dataset}

   To constrain the relevant IDM models (\cite{refId0}), we use the newly revised
   observational $H(z)$ data (OHD)(\cite{Zhang_2014};\cite{PhysRevD.71.123001};
   \cite{Daniel.Stern_2010};\cite{M.Moresco_2012};\cite{Moresco_2016};
   \cite{10.1093/mnras/stx301};\cite{10.1093/mnrasl/slv037};\cite{Borghi_2022};
   \cite{Jiao_2023}),the Pantheon+ set of 1701 SNe Ia (\cite{Scolnic_2022}),
   the BAO data from SDSS and DESI 2024.

\subsection{The observational H(z) data}

   It is widely known that the Hubble parameter $H(z)$ depends on
   the differential age as a function of redshift $z$ in the form
   \begin{equation}
      H(z)=-\frac{1}{1+z}\frac{\mathrm{d}z}{\mathrm{d}t},
   \end{equation}
   which provides a direct measurement on $H(z)$ based on
   $\mathrm{d}z/\mathrm{d}t$.
   OHD measurements have recently been acquired mainly employing
   cosmic chronometers (CC). The CC method is used to provide 33 observational
   data points, which are taken in the redshift range [0.07, 1.965].
   The Table \ref{tab:1} lists the OHD dataset used in this analysis.
   In this case, $\chi^2$ can be defined as
   \begin{equation}
      \chi_{\text{OHD}}^2=\sum_i^{33}\frac{(H_{\text{th}}-H_{\text{data}})^2}{\sigma_i^2}.
   \end{equation}

   \begin{table}[htbp]
      \caption{The OHD dataset}
      \centering
      \begin{tabular}{llll}
         \hline\hline
         $z$ & $H(z)$ & Reference \\
         \hline     
         0.07 & 69$\pm$19.6 & \cite{Zhang_2014} \\
         0.09 & 69$\pm$12 & \cite{PhysRevD.71.123001} \\
         0.12 & 68.6$\pm$26.2 & \cite{Zhang_2014} \\
         0.17 & 83$\pm$8 & \cite{PhysRevD.71.123001} \\
         0.179 & 75$\pm$4 & \cite{M.Moresco_2012} \\
         0.199 & 75$\pm$5 & \cite{M.Moresco_2012} \\
         0.2 & 72.9$\pm$29.6 & \cite{Zhang_2014} \\
         0.27 & 77$\pm$14 & \cite{PhysRevD.71.123001} \\
         0.28 & 88.8$\pm$36.6 & \cite{Zhang_2014} \\
         0.352 & 83$\pm$14 & \cite{M.Moresco_2012} \\
         0.3802 & 83$\pm$13.5 & \cite{Moresco_2016}  \\
         0.4 & 95$\pm$17 & \cite{PhysRevD.71.123001} \\
         0.4004 & 77$\pm$10.2 & \cite{Moresco_2016} \\
         0.4247 & 87.1$\pm$11.2 & \cite{Moresco_2016} \\
         0.4497 & 92.8$\pm$12.9 & \cite{Moresco_2016} \\
         0.47 & 89$\pm$34 & \cite{10.1093/mnras/stx301} \\
         0.4783 & 80.9$\pm$9 & \cite{Moresco_2016} \\
         0.48 & 97$\pm$62 & \cite{Daniel.Stern_2010} \\
         0.593 & 104$\pm$13 & \cite{M.Moresco_2012} \\
         0.68 & 92$\pm$8 & \cite{M.Moresco_2012} \\
         0.75 & 98.8$\pm$33.6 & \cite{Borghi_2022} \\
         0.781 & 105$\pm$12 & \cite{M.Moresco_2012} \\
         0.8 & 113.1$\pm$15.1 & \cite{Jiao_2023} \\
         0.875 & 125$\pm$17 & \cite{M.Moresco_2012} \\
         0.88 & 90$\pm$40 & \cite{Daniel.Stern_2010} \\
         0.9 & 117$\pm$23 & \cite{PhysRevD.71.123001} \\
         1.037 & 154$\pm$20 & \cite{M.Moresco_2012} \\
         1.3 & 168$\pm$17 & \cite{PhysRevD.71.123001} \\
         1.363 & 160$\pm$33.6 & \cite{10.1093/mnrasl/slv037} \\
         1.43 & 177$\pm$18 & \cite{PhysRevD.71.123001} \\
         1.53 & 140$\pm$14 & \cite{PhysRevD.71.123001} \\
         1.75 & 202$\pm$40 & \cite{PhysRevD.71.123001} \\
         1.965 & 186.5$\pm$50.4 & \cite{10.1093/mnrasl/slv037} \\
         \hline    
      \end{tabular}
      \label{tab:1}
   \end{table}

\subsection{Type Ia supernovae}

   SNe Ia have long been used as "standard candles" to give a direct
   measurement of their luminosity distance, and provides strong constraints
   on cosmological parameters. We use the latest Pantheon+ data set of 1701 
   SNe Ia samples(\cite{Scolnic_2022}), which covers the redshift range [0, 2.26].
   
   We use the fiducial SN Ia magnitude ($M_b$) determined from SH0ES 2021 Cepheid 
   host distances(\cite{Riess_2022}), which gives the $\mu_\text{data}$ and constrains $H_0$
   in advance. To eliminate the influence of $M_b$, we give the $\chi^2$
   as \begin{equation}
      \chi_{\text{SNe}}^2=A-\frac{B^2}{C}+\ln\left(\frac{C}{2\pi}\right),
   \end{equation}
   where $A=\sum_{i=1}^{1701}(\mu_{\text{th}}-\mu_{\text{data}})^2/\sigma_i^2,
   B=\sum_{i=1}^{1701}(\mu_{\text{th}}-\mu_{\text{data}})/\sigma_i^2,C=\sum_i^{1701}1/\sigma_i^2$,
   the distance modulus is $\mu=5\log_{10}(d_L/\text{Mpc})+25$, and the
   luminosity distance $d_L$ can be given as a function of redshift $z$
   \begin{equation}
      d_L=(1+z)\int_0^z\frac{c\mathrm{d}z'}{H(z')}.
   \end{equation}
   However, the Eq.(\ref{eq:9}) just need the $H_0$ to caculate the $M_x$
   and we would still use the simple likelihood function as
   \begin{equation}
      \tilde{\chi}_{\text{SNe}}^2=\Delta^{\text{T}}C^{-1}\Delta,
   \end{equation}
   where $\Delta=(\mu_{\text{th}}-\mu_{\text{data}})$ and $C^{-1}$ is the inverse of the
   covariance matrix of the SNe Ia data.

   $B=\Delta^{\text{T}}C^{-1}$ and $C=\text{sum}(C^{-1})$.

\subsection{Quasar}

   The quasar gives a higher redshift

\subsection{Baryon acoustic oscillation}

   The Baryon acoustic oscillation method (BAO) provides a key cosmological probe
   sensitive to the cosmic expansion history with well-controlled systematics.
   We use two BAO data sets from the SDSS(\cite{PhysRevD.103.083533}) and DESI 2024(\cite{desicollaboration2024desi2024vicosmological}),
   which are given at Table \ref{tab:2} and Table \ref{tab:3}, respectively.
   The redshift is up to 2.33 both in the SDSS and the DESI 2024 dataset.

   The $\chi^2$ function for the BAO data is defined as
   \begin{equation}
      \chi_{\text{BAO}}^2=\sum_i\frac{(D_{\text{th}}/r_{\text{d}}-D_{\text{data}}/r_{\text{d}})^2}{\sigma_i^2},
   \end{equation}
   where $D$ refers to $D_{\text{M}}$, $D_{\text{H}}$, or $D_{\text{V}}$, which ara given as
   \begin{eqnarray}
      D_{\text{M}}(z)&=&c\int_0^z\frac{\text{d}z'}{H(z')},\\
      D_{\text{H}}(z)&=&\frac{c}{H(z)},\\
      D_{\text{V}}(z)&=&\left[zD_{\text{M}}^2(z)D_{\text{H}}(z)\right]^{1/3},\\
   \end{eqnarray}
   and $r_{\text{d}}$ is the sound horizon at the drag epoch, which is given as
   \begin{equation}
      r_{\text{d}}=\int_{z_{\text{drag}}}^{\infty}\frac{c_s\text{d}z'}{H(z')}.
   \end{equation}

   However, the Eq.(\ref{eq:8}) just have a stiff point when $z\to\infty$, so the IDM
   model can not give a constraint to the $r_{\text{d}}$ and we just try to use the cross
   parameter $r_{\text{d}}h$ to give the constraints.

   \begin{table}[htbp]
      \caption{The BAO-only dataset from SDSS}
      \centering
      \begin{tabular}{lllll}
         \hline\hline
            $z_{\text{eff}}$ & $D_{\text{M}}/r_{\text{d}}$ & $D_{\text{H}}/r_{\text{d}}$ & $D_{\text{V}}/r_{\text{d}}$ \\
            \hline
            0.15 & & & 4.47$\pm$0.17 \\
            0.38 & 10.23$\pm$0.17 & 25$\pm$0.76 & \\
            0.51 & 13.36$\pm$0.21 & 22.33$\pm$0.58 & \\
            0.7 & 17.86$\pm$0.33 & 19.33$\pm$0.53 & \\
            0.85 & & & $18.33_{-0.62}^{+0.57}$ \\
            1.48 & 30.69$\pm$0.8 & 13.26$\pm$0.55 & \\
            2.33 & 37.6$\pm$1.9 & 8.93$\pm$0.28 & \\
            2.33 & 37.3$\pm$1.7 & 9.08$\pm$0.34 & \\
            \hline
      \end{tabular}
      \label{tab:2}
   \end{table}

   \begin{table}[htbp]
      \caption{The BAO dataset from DESI 2024}
      \centering
      \begin{tabular}{lllll}
         \hline\hline
            $z_{\text{eff}}$ & $D_{\text{M}}/r_{\text{d}}$ & $D_{\text{H}}/r_{\text{d}}$ & $D_{\text{V}}/r_{\text{d}}$ \\
            \hline
            0.295 & & & 7.93$\pm$0.15 \\
            0.51 & 13.62$\pm$0.25 & 20.98$\pm$0.61 & \\
            0.706 & 16.85$\pm$0.32 & 20.08$\pm$0.6 & \\
            0.93 & 21.71$\pm$0.28 & 17.88$\pm$0.35 & \\
            1.317 & 27.79$\pm$0.69 & 13.82$\pm$0.42 & \\
            1.491 & & & 26.07$\pm$0.67 \\
            2.33 & 39.71$\pm$0.94 & 8.52$\pm$0.17 & \\
            \hline
      \end{tabular}
      \label{tab:3}
   \end{table}

\section{Constraint results}

\section{Conclusions}

\bibliographystyle{aa}
\bibliography{reference}

\end{document}